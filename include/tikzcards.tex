\pgfmathsetmacro{\cardwidth}{3.0}
\pgfmathsetmacro{\cardheight}{4.1}


\pgfmathsetmacro{\imagewidth}{\cardwidth}
\pgfmathsetmacro{\imageheight}{0.75*\cardheight}
\pgfmathsetmacro{\stripwidth}{0.7}
\pgfmathsetmacro{\strippadding}{0.05}
\pgfmathsetmacro{\textpadding}{0.1}
\pgfmathsetmacro{\titley}{\strippadding+\textpadding+0.5*\stripwidth}

\usepackage{relsize}
\tikzset{fontscale/.style = {font=\relsize{#1}}}


%   Formen der einzelnen Kartenelemente/-bestandteile

\def\shapeCard{(0,0) rectangle (\cardwidth,\cardheight)} 

\def\shapeLeftStripLong{(\strippadding,-0.2) rectangle (\strippadding+\stripwidth,\cardheight-\strippadding-\strippadding-1)}

\def\shapeLeftStripShort{(\strippadding,\cardheight-\strippadding-1) rectangle (\strippadding+\stripwidth,\cardheight+0.2)}

\def\shapeRightStripShort{(\cardwidth-\stripwidth-\strippadding,\cardheight-\strippadding-1) rectangle (\cardwidth-\strippadding,\cardheight+0.2)}

\def\shapeTitleArea{(3*\strippadding,\strippadding) rectangle (\cardwidth-3*\strippadding,1.25*\stripwidth)}

\def\shapeContentArea{(2*\strippadding+\stripwidth,0.5*\cardheight) rectangle (\cardwidth+0.2,-0.2)}


%   Stylings für die Elemente definieren
\tikzset{
    %   runde Ecken für die Karten
    cardcorners/.style={},
    %   runde Ecken für die "Fähnchen"
    elementcorners/.style={
        rounded corners=0.1cm
    },
    %   Schlagschatten für die "Fähnchen"
    stripshadow/.style={
        drop shadow={
            opacity=.5,
            %shadow,
            color=black
        }
    },
    %   Style für die "Fähnchen"
    strip/.style={
        elementcorners,
        stripshadow
    },
    %   Bild für das Kartenmotiv
    cardimage/.style={
        path picture={
            \node[below=-1.5mm] at (0.5*\cardwidth,\cardheight) {
                \includegraphics[width=\imagewidth cm]{#1}
            };
        }
    },
}

%   TikZ-Raster
\newcommand{\carddebug}{
    \draw [step=1,help lines] (0,0) grid (\cardwidth,\cardheight);
}

%   Rahmen der Karte
\newcommand{\cardborder}{
    \draw[lightgray] (0,0) rectangle (\cardwidth,\cardheight);
}

%   Hintergrund der Karte
\newcommand{\cardbackground}[1]{
    \draw[draw=none, cardcorners, cardimage=#1] \shapeCard;
}

%   Titel der Karte
\newcommand{\cardtitle}[1]{
    %\draw[pattern=soft crosshatch,rounded corners=0.1cm] \shapeTitleArea;
    \fill[elementcorners,titlebg,opacity=.85] \shapeTitleArea;
    \node[text width=3.75cm] at (0.5*\cardwidth,\titley) [fontscale=-2.5] {
        \begin{center}
            \color{white}\uppercase{#1}
        \end{center}
    };
}
